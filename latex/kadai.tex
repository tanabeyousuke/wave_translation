\documentclass[dvipdfmx]{article}
\usepackage{amsmath}
\usepackage{amssymb}
\usepackage[top=25truemm,bottom=20truemm,left=20truemm,right=20truemm]{geometry}
\usepackage{plext} % 日本語の各種設定
\usepackage{tikz}
\usetikzlibrary{arrows.meta}
\begin{document}
\section*{課題研究一学期成果報告書}
今年度の課題研究ではFortranとCを使ってシンセサイザーを作り、演奏することを目標とした。\par
Fortranだけでは音声出力のライブラリが無いので、波形はFortranで生成・加工してCに送り、C側のPulseAudioを利用して音声出力を行う構造を採用した。大まかなフローチャートは以下の通り。\par
\begin{tikzpicture}
  \draw[dashed](7, 0)--(7, -10);
  \node at (3, 0) {C側};
  \node at (11, 0) {Fortran側};
  
  \draw(0, -2)rectangle(6, -3);
  \node at (3, -2.5) {音声サーバ、波形バッファの準備};

  \draw[->] (3, -3)--(3, -4)--(11, -4)--(11, -5);

  \draw(8, -5)rectangle(14, -6);
  \node at (11, -5.5) {波形バッファに音の波形を書き込む};

  \draw[->] (11, -6)--(11, -7)--(3, -7)--(3, -8);

  \draw(0, -8)rectangle(6, -9);
  \node at (3, -8.5) {波形バッファを音声として出力};
\end{tikzpicture}

現在は、シンプルAPIを利用して一秒間正弦波を出力することに成功している。\par

\subsection*{今後の目標}
\subsubsection*{非同期APIを使い、実用的なものにする}
シンプルAPIはあくまで「シンプルなコードで音を出す」のみに限定しており、波形を読み込む$\to$再生しか出来ない。\par
波形の生成に時間がかかる場合音が途切れてしまうので、それを防ぐためにより柔軟な非同期APIを利用する。
\subsubsection*{オシレータ、フィルタ等の実装}
減算方式シンセサイザー(箪笥など)の基本的なパーツは次の通り。
\begin{enumerate}
\item オシレータ
\item フィルタ
\item エンベロープ
\item LFO
\end{enumerate}
これらが次の図のように繋がり、音波を生成する。\par

\begin{tikzpicture}
  \draw(0, 0)rectangle(2, 0.5);
  \node at (1, 0.25) {オシレータ};
  \node at (1, 0.75) {元になる波形を生成};

  \draw[->] (2, 0.25)--(3, 0.25);

  \draw(3, 0)rectangle(5, 0.5);
  \node at (4, 0.25) {フィルタ};
  \node at (4, 0.75) {音色を変化させる};

  \draw[->] (5, 0.25)--(6, 0.25);

  \draw(6, 0)rectangle(10, 0.5);
  \node at (8, 0.25) {LFO, エンベロープ};
  \node at (8, 0.75) {音に強弱をつける};
  
  \draw[->] (10, 0.25)--(11, 0.25);

  \draw(11, 0)rectangle(13, 0.5);
  \node at (12, 0.25) {アンプ};
  \node at (12, 0.75) {音量を調節};
\end{tikzpicture}

\subsubsection*{テキストファイルで演奏できるようにする}
とても重要。アセンブリ言語風のテキストで演奏させる。\par
以下はド、レ、ミの3音を低い方から順に出すテキスト。

\begin{verbatim}
  .globl main
main:
  snd c 4
  snd d 4
  snd e 4
\end{verbatim}

\subsubsection*{トークボックスの実装}
トークボックスとは、スピーカーの音をゴムホースを通して口内に響かせ、口の形を変えることで音色を変化させる楽器。ヴォコーダーと倍音変化が似ている。\par

\subsubsection*{演奏}
最終目標として、YMOの「Technopolis」を演奏する。\par

\end{document}
